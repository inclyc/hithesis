\section{课题来源及研究的目的和意义}

计算科学与高性能计算在当代科学研究和工程应用中扮演着至关重要的角色。
分子动力学模拟作为一种重要的计算科学工具,广泛用于模拟分子体系的行为,如生物分子、聚合物和材料。
其中,NAMD(NAnoscale Molecular Dynamics)\cite{phillips2005scalable}是一款广泛应用的分子动力学软件,其高度并行的特性使其成为模拟大规模分子系统的有力工具。
然而,目前,NAMD在国产CPU上的性能与一些国际竞争对手相比存在一定差距。
因此,本研究旨在探索NAMD分子动力学算法在国产CPU上的优化方法,以提高其性能和效率。

\begin{figure}[h]
    \centering
    \includegraphics{images/namd-logo.png}
    \caption{NAMD 标识}
\end{figure}

本研究的目的在于提高NAMD在国产CPU上的计算性能,为国内科研和工程应用提供更高效的分子动力学模拟工具。
通过优化NAMD算法,我们将探讨如何充分利用国产CPU的硬件资源,提高模拟速度,降低计算成本,并使国内科研机构和工程部门能够更好地应对复杂的分子系统建模问题。

研究的意义在于不仅促进了分子动力学模拟的发展,还有助于推动国产CPU技术在科学计算领域的应用和发展。
这项研究将有助于提高我国在生物医药、材料科学、化学工程等领域的研究和创新水平,为国家自主研发计算机技术提供实际应用的范例。
最终,通过这项研究,我们将为国内高性能计算和科学研究社区提供一种有效的分子模拟工具,为我国的科学研究和工程应用带来巨大的潜在价值。

\section{国内外在该方向的研究现状及分析}

NAMD 是一种开源高性能分子动力学模拟软件,用于模拟原子和分子之间的相互作用以及它们的运动行为。
NAMD的主要应用领域包括生物医学研究\cite{yao2020molecular}、生物化学\cite{knott2020characterization}、药物设计\cite{han2020computational}、材料科学和纳米技术。
这个软件的独特之处在于其能够模拟极小尺度的系统,如生物分子、蛋白质、DNA、聚合物和其他化学体系,以揭示它们的结构、动力学和相互作用。

% 2020 Gorden Bell 用了这个软件

\section{主要研究内容}
\section{研究方案}
\section{进度安排,预期达到的目标}
\section{课题已具备和所需的条件、经费}
\section{研究过程中可能遇到的困难和问题,解决的措施}
\section{主要参考文献}
\bibliographystyle{hithesis}
\bibliography{reference}

% Local Variables:
% TeX-master: "../mainart"
% TeX-engine: xetex
% End:
